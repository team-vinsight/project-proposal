\section{Conclusion}
This research project has successfully addressed the critical challenge of autonomous UAV navigation in GPS-restricted environments by implementing a robust Visual-Inertial Navigation System (VINS). The project's comprehensive literature review established VINS as the premier solution for GPS-denied navigation, highlighting its ability to fuse complementary data from a camera and an Inertial Measurement Unit (IMU) to achieve real-time, high-accuracy state estimation. The review further detailed the foundational principles of these sensors and the core architectural paradigms of VINS, including the superior performance of tightly-coupled and optimization-based approaches over filter-based methods.

The project's methodology, grounded in these findings, outlines a practical and systematic approach to building a reliable VINS. It details the selection of a specific solution, system architecture, hardware and software configuration, and a clear evaluation plan. The research timeline provides a structured, phased approach to achieving the project's objectives, from initial problem formulation and resource gathering to the final stages of drone integration, model improvements, and documentation.

In essence, this work provides a detailed framework for developing and validating a VINS for autonomous UAVs. The successful completion of the proposed research timeline will not only yield a functional navigation system but will also serve as a foundational step toward solving the "last mile" problem of UAV autonomy, enabling drones to safely operate in complex, real-world scenarios where GPS is unreliable. Future work will focus on integrating the improved model into a drone swarm to explore collective autonomy and enhance mission capabilities.